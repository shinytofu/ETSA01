\documentclass[12pt,titlepage]{article}
\usepackage[utf8]{inputenc}
\usepackage[T1]{fontenc}
\usepackage[english]{babel}
\usepackage[a4paper]{geometry}
\frenchspacing
\usepackage{amsfonts}
\usepackage{amsmath}
\usepackage{rotating}
\usepackage{amssymb}
\usepackage{amsthm}
\usepackage{setspace}
\usepackage{fullpage}
\usepackage{tocbibind}
\usepackage{graphicx}
\usepackage{url}
\usepackage{verbatim}
\usepackage{listings}
%\usepackage{gitinfo2}
\usepackage{hyperref}
\usepackage{cleveref}
\usepackage{cite}
\usepackage{color}
\usepackage{enumitem}
\usepackage{nameref}
\usepackage{float}

\definecolor{pblue}{rgb}{0.13,0.13,1}
\definecolor{pgreen}{rgb}{0,0.5,0}
\definecolor{pred}{rgb}{0.9,0,0}
\definecolor{pgrey}{rgb}{0.46,0.45,0.48}
\lstset{language=Java,
  showspaces=false,
  showtabs=false,
  breaklines=true,
  showstringspaces=false,
  breakatwhitespace=true,
  commentstyle=\color{pgreen},
  keywordstyle=\color{pblue},
  stringstyle=\color{pred},
  basicstyle=\ttfamily,
  moredelim=[il][\textcolor{pgrey}]{$$},
  moredelim=[is][\textcolor{pgrey}]{\%\%}{\%\%}
}
\renewcommand{\labelitemi}{$\bullet$}
\renewcommand{\labelitemii}{$\cdot$}
\renewcommand{\labelitemiii}{$\diamond$}
\renewcommand{\labelitemiv}{$\ast$}


\begin{document}
\title{
	Test Plan for Bike Garage Pro (Group 33, 2015)\\
%	\vspace{0.2in}
%	\normalsize Current version (commit hash): \gitAbbrevHash\\
%	\normalsize Version history: \url{http://git.io/vvPe0}
}
\author{
	Alexander Skafte\\
	\url{tfy13ask@student.lu.se}\\
}
\date{}


\maketitle
\newpage
\tableofcontents
\thispagestyle{empty}
\setcounter{page}{0}
\newpage

% ----- REFERENCES & -----------------------------------------------------------
% ----- BIBLIOGRAPHY -----------------------------------------------------------

\section{References}
\label{sec:references}

\begin{itemize}
	\item \textit{Examples and Exercises in the Software
		Engineering Process}. ETSA01 VT 2015. Deparment of Computer
		Science, Lund University. March 10, 2015.
	\item \textit{Software Requirements Specification for Bicycle Garage
		Pro}. ETSA01, Group 33, 2015.
\end{itemize}

% ----- INTRODUCTION -----------------------------------------------------------

\section{Introduction}

\subsection{Project model}

\subsection{Purpose}

The projects aim is to develop software for a bicycle garage, which handles
storage of bicycles. Software will be developed for hardware already specified.

\subsection{Goals}

\subsubsection{Product goals}
The software shall have low deficiency and should handle a high number of users
with good, and optimized, performance. The product will take care of users
passing through the entrance with their bicycle - the user will enter a pincode
and scan a barcode for their bicycle to store it. The user can then exit the
garage, and to exit with a bicycle the same procedure will be done as the one
to store it.

\subsubsection{Business goals}
A business wants to offer cheap and reliable bicycle storage to their customers,
whom may want to protect their bicycle against misfortune.

\subsection{Limitations}
The project uses specific hardware, which will limit be the limit for security
and performance if the software is properly optimized and functioning.The date
for delivery is already specified and thus tests with good coverage needs to be
implemented. The project has no funding, therefor the manpower available is what
the project is limited by.

% ----- PROJECT ORGANIZATION ---------------------------------------------------

\section{Project Organization}

\subsection{Development organization}

\textit{\textbf{TODO:} This division of tasks was made afterward... Should we
still keep it?} \\

Responsible for for this project are:

\begin{description}
	\item[Alexander Skafte:] \hfill
		\begin{itemize}
			\item Requirements specification
			\item Test plan
			\item Design document
			\item Project plan
			\item Various reviews
			\item Software development
		\end{itemize}
	\item[Dennis Jin:] \hfill
		\begin{itemize}
			\item Requirements specification
			\item Test plan
			\item Design document
			\item Various reviews
		\end{itemize}
	\item[Petter Berntsson:] \hfill
		\begin{itemize}
			\item Project plan
		\end{itemize}
	\item[Emelie Löthman:] \hfill
		\begin{itemize}
			\item \ldots
		\end{itemize}
	\item[Adam Mrozek:] \hfill
		\begin{itemize}
			\item \ldots
		\end{itemize}
\end{description}

\subsection{Stakeholders}

\begin{itemize}
	\item The municipality
	\item ACME
\end{itemize}

% ----- HARDWARE AND SOFTWARE RESOURCES  ---------------------------------------

\section{Hardware and software resources}
\label{sec:hardware-and-software-resources}

\begin{description}
	\item[\LaTeX:] Typesetting software; used to typeset all documents
		related to the development of \textit{Bicycle Garage Pro}.
	\item[Eclipse:] Integrated desktop environment for software development.
	\item[The Java Programming Language:] The programming language used to
		create the software.
	\item[Google Drive:] Cloud storage reachable for all members to edit
	\item[Git \& GitHub:] For sharing code and \LaTeX~source text
\end{description}

% ----- DIVISION OF LABOR ------------------------------------------------------

\section{Division of labor}

\textit{\textbf{TODO:} Same as ''Development organization''?} \\

\subsection{Activities}

\begin{description}
	\item[Project plan] \hfill \\
		The planning of the project, also this rapport.
	\item[Requirements specifications] \hfill \\
		The requirements for the project are identified and defined.
		This results in a requirements specification.
	\item[Test plan] \hfill \\
		The planning of all tests that will be performed.
		This results in a test plan.
	\item[Design] \hfill \\
		A general description of the systems structure.
		This results in a design document.
		Different people are responsible for different parts of the code.
	\item[Implementation and unit testing] \hfill \\
		Implementation and testing of all parts of the system.
		This is performed according to the design.
	\item[Integration] \hfill \\
		This part is the finishing part of the program,
		which leads to a fully functioning program.
	\item[System test] \hfill \\
		The system is tested in its entirety. This takes part in a
		testing environment which mimics the environment that the
		program will be used in.
\end{description}

\subsection{Deliverables}

\begin{description}
	\item[Project plan] \hfill \\
	\item[Requirements specification] \hfill \\
	\item[Design document] \hfill \\
	\item[Test plan] \hfill \\
	\item[Source code] \hfill \\
\end{description}

\subsection{Landmarks}

???

\subsection{Schedule and estimated work}

% ----- REPORT, FOLLOW-UP AND QUALITY ASSURANCE --------------------------------

\section{Report, follow-up and quality assurance}

% TODO: Translate!

%För att alla alltid ska ha tillgång till den senaste versionen av källkoden så
%använder vi oss av google code och subversion används som
%versionshanteringssystem. Det finns en Wiki att tillgå, för information som inte
%direkt ingår i projektet. Latex källfiler hanteras på samma sätt som källkoden.
%På projektets googlesida finns möjlighet att buggrapportera, detta kommer att
%användas för både källkod och dokumentfiler. I första hand är det
%dokumentansvarige som ska rätta till felet. Man kan också se vem som har ändrat
%vad och vid vilken tidpunkt ändringen har skett. Det går även att gå tillbaka
%till äldre versioner. Färdiga dokument laddas upp i pdf-format under Downloads.
%
%Varje onsdag har vi ett planerat veckomöte. Syftet med mötet är att gå igenom
%vad som har gjorts under föregående vecka och planera inför kommande vecka
%utifrån tidsplanen. Projektledaren är ordförande under dessa möten och håller i
%dagordningen. Färdiga dokument godkänns även under mötet.
%
%Då projektet är litet är det inga problem att kommunicera med e-mail och
%telefon. Dessutom träffas projektmedelmmarna dagligen.
%
%Samtliga dokument ska kontrolläsas av alla projektdeltagare och godkännas, det
%slutliga godkännandets utförs av den som har ansvaret för dokumentet.
%
%Det kommer ske en inspektion av projekthandledaren vid varje inlämning som finns
%ovan i kapitel 6.4.

% ----- RISK ANALYSIS ----------------------------------------------------------

\section{Risk analysis}

\textit{\textbf{TODO:} Add more risk cases.} \\

Hardware noncompliance
Risk; Low
Effect; Devastating
How do we fix it? Contact the hardware developers and ask for new hardware specifications
Riskindicators; The appliance does not work properly, even if the virtual testing passed.

% ----- APPENDIX ---------------------------------------------------------------
% ------------- Test Cases -----------------------------------------------------

\newpage
\appendix

\section{TODO: Any appendices?}
\label{app:TODO}

\end{document}

