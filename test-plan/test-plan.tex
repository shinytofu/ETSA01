\documentclass[12pt,titlepage,bibliography=totoc]{article}
\usepackage[utf8]{inputenc}
\usepackage[T1]{fontenc}
\usepackage[english]{babel}
\usepackage[a4paper]{geometry}
\frenchspacing
\usepackage{amsfonts}
\usepackage{amsmath}
\usepackage{rotating}
\usepackage{amssymb}
\usepackage{amsthm}
\usepackage{setspace}
\usepackage{fullpage}
\usepackage{tocbibind}
\usepackage{graphicx}
\usepackage{url}
\usepackage{verbatim}
\usepackage{listings}
%\usepackage{gitinfo2}
\usepackage{hyperref}
\usepackage{cleveref}
\usepackage{cite}
\usepackage{color}
\usepackage{enumitem}
\usepackage{nameref}
\usepackage{float}

\definecolor{pblue}{rgb}{0.13,0.13,1}
\definecolor{pgreen}{rgb}{0,0.5,0}
\definecolor{pred}{rgb}{0.9,0,0}
\definecolor{pgrey}{rgb}{0.46,0.45,0.48}
\lstset{language=Java,
  showspaces=false,
  showtabs=false,
  breaklines=true,
  showstringspaces=false,
  breakatwhitespace=true,
  commentstyle=\color{pgreen},
  keywordstyle=\color{pblue},
  stringstyle=\color{pred},
  basicstyle=\ttfamily,
  moredelim=[il][\textcolor{pgrey}]{$$},
  moredelim=[is][\textcolor{pgrey}]{\%\%}{\%\%}
}
\renewcommand{\labelitemi}{$\bullet$}
\renewcommand{\labelitemii}{$\cdot$}
\renewcommand{\labelitemiii}{$\diamond$}
\renewcommand{\labelitemiv}{$\ast$}


% TODO:
%	Every subtitle will have the following stuff:
%		- Performed by:
%		- Type of test:
%		- Criterion
%		- Stop rule

% TODO:
%	Remove empty newlines at the bottom of this document :)

\begin{document}
\title{
	Test Plan for Bike Garage Pro (Group 33, 2015)\\
%	\vspace{0.2in}
%	\normalsize Current version (commit hash): \gitAbbrevHash\\
%	\normalsize Version history: \url{http://git.io/vvPe0}
}
\author{
	Alexander Skafte\\
	\url{tfy13ask@student.lu.se}\\
}
\date{}


\maketitle
\newpage
\tableofcontents
\thispagestyle{empty}
\setcounter{page}{0}
\newpage

% ----- REFERENCES -------------------------------------------------------------

\section{References}
\label{sec:references}

% ----- INTRODUCTION -----------------------------------------------------------

\section{Introduction}
\label{sec:introduction}
\subsection{Tested system}

The system described in this document is the software for a public bicycle
garage. This software is responsible for managing the authentication of users
and the management of their information and their bicycles.

This document provides a specification for testing the bicycle garage software.
The test process consists of the following phases:

\begin{itemize}
	\item Unit testing
	\item Integration testing
	\item System testing
	\item Acceptance testing
\end{itemize}

% ----- TEST PROCESS -----------------------------------------------------------

\section{Test process}
\label{sec:test-process}
\subsection{Process overview}


\subsection{Unit testing}
\label{subsec:unit-testing}

Every non-trivial function is tested in software through the use of a test suite
library.

\begin{description}
	\item[Performed by:]	Developers
	\item[Type of test:]	Structural
	\item[Criteria:]	Every line of code is tested
	\item[Stop rule:]	No errors found
\end{description}


\subsection{Integration testing}
\label{subsec:integration-testing}

Integration testing is performed in a similar way to unit testing, but larger
and more inclusive modules are tested. Each module is tested in software through
the use of a test suite library.

\begin{description}
	\item[Performed by:]	Developers
	\item[Type of test:]	Structural
	\item[Criteria:]	Every API method is tested completely
	\item[Stop rule:]	No errors found
\end{description}


\subsection{System testing}
\label{subsec:system-testing}

During system testing, all requirements specified inside the Software
Requirements Specification are tested.

\begin{description}
	\item[Performed by:]	Developers
	\item[Type of test:]	Functional
	\item[Criteria:]	All requirements inside the SRS are fulfilled
	\item[Stop rule:]	No critical errors found
\end{description}

\subsection{Acceptance testing}

Acceptance testing is performed by the client and not the developers, and is
therefore not discussed in this document.

% ----- TESTED ITEMS -----------------------------------------------------------

\section{Tested items}
\label{sec:tested-items}

%
% TODO TODO TODO TODO TODO TODO TODO TODO TODO TODO TODO TODO TODO TODO TODO
%

% ----- TEST RECORDING PROCEDURE -----------------------------------------------

\section{Test recording procedure}
\label{sec:test-recording-procedure}
\subsection{Unit testing}
\subsection{Integration testing}
\subsection{System testing}
\subsection{Acceptance testing}

% ----- TEST CASES FOR SYSTEM TESTING ------------------------------------------

\section{Test cases for system testing}
\label{sec:test-cases-for-system-testing}
\subsection{Test cases}
\subsection{Requirements coverage and traceability}

\bibliography{bibliography}{}
\bibliographystyle{plain}

% TODO: Fix this shit later
%\appendix

\section{Test cases}
\label{app:test-cases}

\subsection{Test case 1}



% ----- APPENDIX ---------------------------------------------------------------

\newpage
\appendix

\section{Test cases}
\label{app:test-cases}

\begin{usecase}
	\addtitle{(Template) Use case 0:}{Someone wants to...}
	\addfield{Primary actor:}{Someone}
	\addfield{Preconditions:}{Something is something}
	\addfield{Postconditions:}{Something is something else}
	\addscenario{Main success scenario:}{
		\item First item is something fun
		\item Second item is something else (maybe fun, maybe not)
		\item Third item is always fun, because now we're done
	}
\end{usecase}


\end{document}


















